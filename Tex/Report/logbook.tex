\documentclass[11pt, a4paper]{article} % A4 størrelse
\usepackage{amsmath}
\usepackage{amssymb}
\usepackage{bm}
\usepackage[usenames,dvipsnames]{color}
\usepackage{enumerate}

%
% Useful definitions for vectors and stuff
%

% matrices and vectors
\def\A{{\bf A}}
\def\a{{\bf a}}
\def\B{{\bf B}}
\def\b{{\bf b}}
\def\C{{\bf C}}

\def\D{{\bf D}}
\def\E{{\bf E}}
\def\e{{\bf e}}

\def\F{{\bf F}}
\def\f{{\bf f}}
\def\G{{\bf G}}
\def\g{{\bf g}}
\def\H{{\bf H}}
\def\h{{\bf h}}
\def\I{{\bf I}}



\def\K{{\bf K}}
\def\k{{\bf k}}
\def\L{{\bf L}}

\def\M{{\bf M}}
\def\m{{\bf m}}




\def\P{{\bf P}}

\def\Q{{\bf Q}}
\def\q{{\bf q}}
\def\R{{\bf R}}
\def\r{{\bf r}}
\def\S{{\bf S}}
\def\s{{\bf s}}
\def\t{{\bf t}}
\def\U{{\bf U}}
\def\u{{\bf u}}
\def\V{{\bf V}}
\def\v{{\bf v}}
\def\W{{\bf W}}
\def\w{{\bf w}}
\def\X{{\bf X}}
\def\x{{\bf x}}
\def\Y{{\bf Y}}
\def\y{{\bf y}}
\def\Z{{\bf Z}}
\def\z{{\bf z}}

% permutations
\def\PL{{\bf P}_\mathrm{r}}
\def\PR{{\bf P}_\mathrm{c}}
\def\PC{{\bf P}_\mathrm{f}}

% more stuff
\def\BNet{{\cal B}}
\def\GR{{\cal D}}
\def\ED{{\cal E}}
% \def\S{{\cal S}}
\def\VR{{\cal V}}
\def\Sp{{\cal S}}
\def\0{{\bf 0}}
\def\1{{\bf 1}}

\def\Re{\mathbb{R}}
\def\Ex{\mathbb{E}}
\def\Lik{\mathcal{L}}

\def\<{\, \langle \,}
\def\>{\, \rangle \,}
\def\inv{^{-1}}
\def\ts{^\top}
\def\lik{{\cal L}}
\def\topc{\mathrm{top}}

% mathrms
\def\sign{\mathrm{sign}}
\def\max{\mathrm{max}}
\def\min{\mathrm{min}}
\def\tr{\mathrm{trace}}
\def\diag{\mathrm{diag}}
\def\MAP{\mathrm{MAP}}
\def\ICA{\mathrm{ICA}}
\def\FM{\mathrm{FM}}
\def\DAG{\mathrm{DAG}}
\def\FA{\mathrm{FA}}
\def\BN{\mathrm{BN}}
\def\rep{\mathrm{rep}}
\def\block{\mathrm{block}}
\def\abs{\mathrm{abs}}

% bold greeks
\def\balpha{\bm{\alpha}}
\def\bbeta{\bm{\beta}}
\def\bPsi{\bm{\Psi}}
\def\bSigma{\bm{\Sigma}}
\def\bepsilon{\bm{\epsilon}}
\def\bupsilon{\bm{\upsilon}}
\def\bmeta{\bm{\eta}}
\def\bmu{\bm{\mu}}
\def\bnu{\bm{\nu}}
\def\btau{\bm{\tau}}
\def\bLambda{\bm{\Lambda}}
\def\bTheta{\bm{\Theta}}

% distributions
\def\DN{\mathcal{N}}
\def\TN{\mathcal{TN}}
\def\IG{\mathrm{IG}}
\def\La{\mathrm{Laplace}}
\def\Exp{\mathrm{Exponential}}
\def\Uni{\mathrm{Uniform}}
\def\Be{\mathrm{Beta}}
\def\Bin{\mathrm{Binomial}}
\def\Ber{\mathrm{Bernoulli}}
\def\Ga{\mathrm{Gamma}}
\def\Poi{\mathrm{Poisson}}
\def\GP{\mathrm{GP}}
\def\DP{\mathrm{DP}}
\def\TP{\mathrm{TP}}
\def\IGamma{{\cal IG}}
\def\DGamma{{\cal G}}
\def\Wish{\mathrm{Wishart}}
\def\IW{{\cal IW}}
\def\MOD{\mathcal{M}}
\def\LN{\mathrm{logNormal}}

% commands
\newcommand{\argmax}{\operatornamewithlimits{argmax}}
\newcommand{\argmin}{\operatornamewithlimits{argmin}}
\newcommand{\deff}{\overset{\underset{\mathrm{def}}{}}{=}}
\newcommand{\indi}{\overset{\underset{\mathrm{ind}}{}}{\sim}}
\newcommand{\Perp}{\perp \! \! \! \perp}

\newcommand{\ie}{i.e. }
\newcommand{\eg}{e.g. }
\newcommand{\wrt}{w.r.t. }
\newcommand{\iid}{i.i.d. }

\newcommand{\tmp}[1]{{\color{orange} {\bf #1}}}
\newcommand{\quotes}[1]{``#1''}

\def\ctilde{\kern -.04em\lower .7ex\hbox{\~{}}\kern .04em}

\def\Tiny{\fontsize{5pt}{5pt}\selectfont}


\title{Logbook}

\begin{document}
\maketitle

\section*{Logbook}


\subsection*{General stuff}



\subsection*{Week 1 (9): 24.02.2014 - 02.03.2014}
\subsubsection*{Project meeting}
No project meeting was possible this week, and we had yet to decided between 
\begin{enumerate}
\item Randomized algorithms: \\
\emph{A Statistical Perspective on Algorithmic Leveraging, Ping Ma, Micheal W. Mahoney, Bin Yu}.  $http://arxiv.org/abs/1306.5362$
\item Spectral learning of HMMs:\\
\emph{A Method of Moments for Mixture Models and Hidden Markov Models}. A. Anandkumar, D. Hsu, and S.M. Kakade. Preprint, Feb. 2012 : $http://newport.eecs.uci.edu/anandkumar/pubs/AnandkumarEtal_mixtures12.pdf$
\end{enumerate}

We spend the week getting an overview of the articles and the projects.

\subsection*{Week 2 (10): 03.03.2014 - 09.03.2014}
\subsubsection*{Project meeting}
\textbf{Questions:}\\
\begin{itemize}
\item What is the idea behind leveraging for least-squares regression?
\item Can we generalise the idea?
\item Can leveraging improve performance in video screen classification?
\item Video classification e.g. faces, emotions, gender.
\end{itemize}

\textbf{Implementation:}\\
No implementation at this point.

\textbf{Results:}\\
No results at this time.

\textbf{Decisions:}\\
To work with Randomized algorithms.
\\
To gain a better understanding of the underlying idea of leveraging, by watching a talk on \emph{Statistical Leverage and Improved Matrix Algorithms} by M. W. Mahoney (\emph{http://videolectures.net/icml09\_mahoney\_itslima/}).

\subsubsection*{Updated Project Goals and Delimitation}
\begin{itemize}
\item Validation of the results shown by Ma. et al.
\item Can we generalize the idea of leveraging for a general likelihood function?
\end{itemize}

\subsection*{Week 3 (11): 10.03.2014 - 16.03.2014}
\subsubsection*{Project meeting}
\textbf{Questions:}\\
\begin{itemize}
\item How does the leverage scores look for LS-regression? (Plotting $H_{n,n}$ vs. $||x_n||$)
\end{itemize}
\textbf{Implementation:}\\
No implementation at this point.

\textbf{Results:}\\
\begin{itemize}
\item The general idea of leveraging is to identify how the estimated value $\hat{\y}$ relates to the targeted value $\y$. Which for LS-regression is $\hat{\y} = H \y$.
\end{itemize}

\textbf{Decisions:}\\

\subsubsection*{Updated Project Goals and Delimitation}
\begin{itemize}
\item Can we generalise the expression $\hat{\y} = H \y$ to logistic regression?
\end{itemize}

\subsection*{Week 4 (12): 17.03.2014 - 23.03.2014}
\subsubsection*{Project meeting}
\textbf{Questions:}\\
\begin{itemize}
\item How are the distributions used by Ma. et al. calculated?
\item Finding emotional faces datasets.
\end{itemize}

\textbf{Implementation:}\\
\begin{itemize}
\item Finding leverage-scores for LS-regression
\item Solving LS-regression when comparing uniform- to leverage-based sampling.
\item Illustrating leverage scores ($H_{n,n}$ vs. $||x_n||$)
\end{itemize}

\textbf{Results:}\\
Initial results promising, but only single run performance between uniform- and leverage-based sampling.

\textbf{Decisions:}\\

\subsubsection*{Updated Project Goals and Delimitation}
\begin{itemize}
\item We want to validate the results of Ma et. al. empirically.
\item In video classification we want to do binary classification of \emph{happy} and \emph{sad} faces.
\end{itemize}


\subsection*{Week 5 (13): 24.03.2014 - 30.03.2014}
\subsubsection*{Project meeting}
\textbf{Questions:}\\
\begin{itemize}
\item Will using the leverage-scores for LS-regression improve our performance in binary classification?
\end{itemize}

\textbf{Implementation:}\\
\begin{itemize}
\item The three distributions $GA$,$T3$ and $T1$ are implemented, and tested for linear regression.
\item Learning curves and test-framework for LS-regression, used for testing the results show by Ma et al.
\item Three distributions for binary classification data, also named $GA$, $T3$ and $T1$ which represent respectively classification data with nearly uniform, moderately non-uniform and very non-uniform leverage-scores.
\end{itemize}

\textbf{Results:}\\
\begin{itemize}
\item We get comparable results on LS-regression to those shown by Ma et al.
\item A leverage-based sampling does not improve for GA-type data, as the leverage scores are approximately uniform, thus there are no "important" datapoints that can be sampled.
\item A leverage-based sampling for T3-type data consistently performs better or equal to a uniform sampling. Although the performance increase modest.
\item A leverage-based sampling for T1-type data also consistently outperforms a uniform-based sampling, this is expected as the T1 data have very non-uniform leverage scores i.e. "important" datapoints. 
\end{itemize}

\textbf{Decisions:}\\
Generalisation of the leverage-based sampling scheme $\frac{\delta \hat{\y}}{\delta \y}$ to logistic regression, as well as a sampling distribution based on the uncertainty of the predictions (asymptotic theory) is to be done by Lars Kai.

\subsubsection*{Updated Project Goals and Delimitation}
\begin{itemize}
\item Will using the leverage-scores for LS-regression improve our performance in binary classification?
\item We have validated the results of Ma et al. for LS-regression on $GA$,$T3$ and $T1$ distributed data.
\end{itemize}
\subsection*{Week 6 (14): 31.03.2014 - 06.04.2014}
\subsubsection*{Project meeting}
%\textbf{Questions:}\\

\textbf{Implementation:}\\
\begin{itemize}
\item Learning curves for logistic regression based on uniform or LS-regression leverage-scores.
\end{itemize}

\textbf{Results:}\\
Initial results using leverage-scores based on LS-regression shows no improvement on GA-type (expected) and performs significantly worse on T3- and T1-type data.

Lars Kai has derived a generalised expression $\frac{\delta \hat{\y}}{\delta \y}$ for a general likelihood function. As well as the uncertainty based sampling approach.

\textbf{Decisions:}\\
Lars Kai gathers his scribles on the back of some insignificant article in a form that is easier to read and follow.

Our full focus is now on midterm preparation.

\subsubsection*{Updated Project Goals and Delimitation}
\begin{itemize}
\item Compare uniform sampling to a leverage based distribution (generalisation) and a uncertainty based distribution.
\end{itemize}

\subsection*{Week 7 (15): 07.04.2014 - 13.04.2014}
\subsubsection*{Project meeting}
Discussion about the midterm and improvements that should be done.

%\textbf{Questions:}\\

%\textbf{Implementation:}\\

%\textbf{Results:}\\

%\textbf{Decisions:}\\

%\subsubsection*{Updated Project Goals and Delimitation}



\subsection*{Week 8 (16): 14.04.2014 - 20.04.2014}
Easter, no project meeting, but sporadic work was done, mostly clarification and bug-correction.


\subsection*{Week 9 (17): 21.04.2014 - 27.04.2014}\subsubsection*{Project meeting}
%\textbf{Questions:}\\

%\textbf{Implementation:}\\

\textbf{Results:}\\
Lars Kai gives us a copy and explains the general concepts behind the generalisation of $\frac{\delta \hat{\y}}{\y}$ and uncertainty-based sampling.

\textbf{Decisions:}\\
We are to understand and digitalise the results derived.

%\subsubsection*{Updated Project Goals and Delimitation}



\subsection*{Week 10 (18): 28.04.2014 - 04.05.2014}
\subsubsection*{Project meeting}
\textbf{Questions:}\\

%\textbf{Implementation:}\\

\textbf{Results:}\\

\textbf{Decisions:}\\

%\subsubsection*{Updated Project Goals and Delimitation}



\subsection*{Week 11 (19): 05.05.2014 - 11.05.2014}
\subsubsection*{Project meeting}
\textbf{Questions:}\\
\begin{itemize}
\item How to illustrate the thought process behind our problem and  approach. Simply illustrative model.
\end{itemize}

\textbf{Implementation:}\\
\begin{itemize}
\item Sensitivity sampling implemented
\item Sensitivity sampling compared to uniform on logistic regression
\end{itemize}

\textbf{Results:}\\
\begin{itemize}
\item Sensitivity sampling "works" on $GA$ and $T3$ data, but the program sometimes crashes on $T1$ data due to overflow errors when calculating the leverage scores.
\item Sensitivity sampling exhibits weird behaviour as the error increases with sampling size.
\end{itemize}

\textbf{Decisions:}\\
\begin{itemize}
\item Debugging sensitivity sampling is postponed, and focus is put on creating an initial poster which can also be used for \emph{Vision Day}.
\end{itemize}

\subsubsection*{Updated Project Goals and Delimitation}

\subsection*{Week 12 (20): 12.05.2014 - 18.05.2014}
\subsubsection*{Project meeting}
\textbf{Questions:}\\
\begin{itemize}
\item What is the effect of weight and unweighed logistic regression?
\end{itemize}

\textbf{Implementation:}\\
\begin{itemize}
\item Generating good illustrations of the problem, process and results.
\end{itemize}

\textbf{Results:}\\
\begin{itemize}
\item The results go from really crappy to as good as uniform.
\end{itemize}

\textbf{Decisions:}\\

%\subsubsection*{Updated Project Goals and Delimitation}

\subsection*{After poster exam: 21.05.2014 - 03.06.2014}
\subsubsection*{Project meeting}
\textbf{Questions:}\\
\begin{itemize}
\item Will a \emph{soft-max} transformation of the leverage-scores improve sensitivity/uncertainty sampling schemes?
\end{itemize}

\textbf{Implementation:}\\
\begin{itemize}
\item Bug-finding and elimination.
\item Minimizing redundant code.
\end{itemize}

\textbf{Results:}\\
\begin{itemize}
\item A soft-max transform improves both sensitivity and uncertainty sampling performance but not significantly compared to uniform sampling. 
\end{itemize}

\textbf{Decisions:}\\
\begin{itemize}
\item Hmm
\end{itemize}

\subsubsection*{Updated Project Goals and Delimitation}



\end{document}
