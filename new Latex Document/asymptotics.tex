\section{Randomised algorithm}
\emph{Uncertianty based on asymptotic liklihood and $\overline{w}$-distribution}\\
Let $\mathcal{L}_\infty$ be the log-likelihood function for a distribution, now let $\mathcal{L}_N$ denote the log-likelihood function based on $N$ observations from this distribution. Furthermore, let $N$ be a large number, for which $L_N \approx L_\infty$. 

\begin{equation}
\mathcal{L_N} = \frac{1}{N} \sum_{n=1}^N \ell_n \qquad \overline{w} \, s.t. \, \frac{\delta \mathcal{L}}{\delta \overline{w}} = \overline{0} \label{eq:ra1}
\end{equation}
Where $\ell_n$ is the log-likelihood of the $n^{th}$ observation. And $\overline{w}$ is the optimal \textbf{(true?)} weights for the distribution, then we combine the expressions from \eqref{eq:ra1}, such that for the optimal weights the following must be fulfilled:

\begin{equation}
\frac{1}{N} \sum_{n = 1}^N \frac{\delta \ell_n}{\delta \overline{w}} = 0 \label{eq:ra2}
\end{equation}

\emph{(Skal vi lige skrive lidt om at $\Delta w = w-w_0$ og er en lille forskydelse i vægtene? Eller er det en lille forskydelse?) }
For each of the $N$ observations, we can write the log-likelihood of the $n^{th}$ observation as:

\begin{equation}
\ell_n(\Delta \overline{w}) = 
\ell_n\left(\overline{w}_0 \right) + 
\left. \frac{\delta \ell_n}{\delta \overline{w}} \right|_{\overline{w}_0} \Delta \overline{w} + 
\frac{1}{2} Tr \left[ \left. \frac{\delta \ell_n}{\delta \overline{w} \delta \overline{w}^T} \right|_{\overline{w}_0} \Delta \overline{w} \Delta \overline{w}^T \right] \label{eq:loglihood-oneObs}
\end{equation}
Or for the entire log-likelihood function:
\begin{equation}
\mathcal{L}_N(\Delta \overline{w}) = 
\mathcal{L}_\infty\left(\overline{w}_0\right) + 
\left(\left.\frac{\delta \mathcal{L}_N}{\delta \overline{w}}\right|_{\overline{w}_0} \right)^T \cdot \Delta \overline{w} +
\frac{1}{2} \Delta \overline{w}^T \left( \left. \frac{\delta^2 \mathcal{L}_N}{\delta \overline{w} \delta \overline{w}^T}\right|_{\overline{w}_0} \right) \Delta \overline{w} + R \label{eq:entireLogLike}
\end{equation}
Where $R$ is the error of the approximation. Furthermore, we define the functions $\overline{\overline{H}}_N = \left. \frac{\delta^2 \mathcal{L}_N}{\delta \overline{w} \delta \overline{w}^T}\right|_{\overline{w}_0}$, and $\overline{g} = \left.\frac{\delta \mathcal{L}_N}{\delta \overline{w}}\right|_{\overline{w}_0}$. And evaluate the condition on $\overline{w}$, stated in \eqref{eq:ra1}:
\begin{equation}
\frac{\delta \mathcal{L_N}}{\delta \overline{w}} = \overline{g}_N+\overline{\overline{H}}_N \Delta \overline{w} = \overline{0}
\end{equation}
We replace $\Delta \overline{w}$ with $\hat{\Delta \overline{w}}$  as $N$ is a finite number, thus only approximating $\Delta \overline{w}$. Isolating $\hat{\Delta \overline{w}}$, and using Ljung [REFERENCE?], we get:
\begin{equation}
\hat{\Delta \overline{w}} = - \overline{\overline{H}}_N^{-1} \cdot \overline{g}_N \overbrace{=}^{Ljung} -\overline{\overline{H}}
_0^{-1} \cdot \overline{g}_{\overline{w}} \left( \overline{w}_0 \right)
\end{equation}
\emph{(Forklaring af at $H_0$ er uafhængig af datasæt, mens $g$ nu er afhængig af $w$ evalueret i $w_0$)}
Besides getting an estimate for $\hat{\Delta \overline{w}}$, we can find the mean of the distribution:
\[
\left\langle \hat{\Delta \overline{w}} \right\rangle = - \overline{\overline{H}}_0^{-1} \left\langle \overline{g}_w \right\rangle (\overline{w}_0)
\]

\subsection*{Nu skal vi prøve at kombinere flere datasets}

